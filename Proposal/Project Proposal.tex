\documentclass[12pt, letterpaper]{article}
\usepackage{fancyhdr}
\usepackage[margin=1in]{geometry}
% Use indent for all paragraphs
\usepackage{indentfirst}



\begin{document}

\begin{center}
    \Large \textbf{Proposal: Solving ODEs in Kinetics Using Artificial Neural Network \\}
    \vspace{0.5em}
    \normalsize Gabriel S. Gusm\~{a}o\textsuperscript{\textdagger}, Zhenyi Yu\textsuperscript{\textdagger}, Nicole (Yuge) Hu\textsuperscript{\textdagger}\\ 
    \vspace{0.2em}
    \textsuperscript{\textdagger}Department of Chemical and Biomolecular Engineering \\ Georgia Institute of Technology \\
    \vspace{0.2em}
    June 9th, 2019 \\
\end{center}

Ordinary differential equations (ODEs) and partial differential equations define numerous questions in the field of chemical engineering spanning from kinetics to transport phenomena. ODEs can be further classified as boundary value problems (BVP) and initial value problems (IVP) in different contexts. Depending on the specific equations, ODEs might not always have explicit analytical solutions, and therefore numerical methods are developed to calculate final outcomes given the problem can be expressed in a standard form. In this project, we would like to explore the possibility of using artificial neural network (ANN) to approach ODEs in the context of kinetics. 

Kinetics are defined as 


\end{document}